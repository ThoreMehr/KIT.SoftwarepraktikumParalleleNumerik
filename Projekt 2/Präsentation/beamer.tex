\documentclass[german,notes,18pt]{beamer}
\mode<presentation>
%\usepackage[ngerman, english]{babel}
\usepackage[utf8]{inputenc}
\usepackage[T1]{fontenc}
\usepackage{amsmath,amsfonts,amssymb}
\usepackage{listings}
\usepackage{eurosym}
\usepackage{multirow}
\usepackage{color}
\usepackage{pdfpages}
\usepackage{graphicx}
\usepackage{wasysym}

\usepackage{tabularx}
\usepackage{color}
\usepackage{colortbl}
\usepackage{graphicx,import}
\usepackage{siunitx}
\usepackage[numbers]{natbib}
\usepackage{multirow}
\graphicspath{{Bilder/}}
%\beamertemplatenavigationsymbolsempty

\usetheme{kit}

\title{Projekt 2}
\subtitle{Arbeiten mit CUDA}
\author{Thore Mehr, Fabian Miltenberger, Sébastien Thill}
\date{04.02.2017}
\institute{Lehrstuhl für Rechnerarchitektur und Parallelverarbeitung (ITEC)}

\definecolor{kit}{HTML}{009682}
\definecolor{darkgreen}{RGB}{0, 180, 0}
\definecolor{darkred}{RGB}{180, 0, 0}
\newcommand{\pro}{$\oplus$}
\newcommand{\contra}{$\ominus$}

\definecolor{shinygray}{RGB}{245,245,245}
\lstset{language=Java,
		keywordstyle=\color{blue}\bfseries,
		numberstyle=\color{blue},
		numbers=left,
		tabsize=4,
		xleftmargin=3.5em,
		xrightmargin=2em}

\begin{document}
	\selectlanguage{ngerman}
	
	\frame{\titlepage}
	
	\section{Gliederung}
	\begin{frame}
		\frametitle{Gliederung}
		
		\begin{enumerate}
			\item Aufgabe 1
			\item usw.
		\end{enumerate}
	\end{frame}

	\section{OpenMP und Tools}
	\subsection{Aufgabe 1}
	\begin{frame}
		\frametitle{Aufgabe 1 -- OpenMP}
		PI\\
		\begin{itemize}
			\item Reihnfolge der Threads nicht determinitisch
			\item Per Crital parallel langsamer als sequenziel
			\item Mit großen N und reduction Speedup nahe Kernzahl
		\end{itemize}
		Mandelbrot\\
		\begin{itemize}
			\item Speedup kleiner
			\item wegen Speicherverwaltung und Caches
		\end{itemize}
	\end{frame}

	
	\section{Fazit}
	\subsection{Fazit}
	\begin{frame}
		\frametitle{Fazit}
		
		\LARGE
		\centering
		Mit \emph{CUDA} lässt es sich parallelisieren.
	\end{frame}

\end{document}