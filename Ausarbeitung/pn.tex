\documentclass[runningheads]{llncs}

%---- Sonderzeichen-------%
\usepackage {ngerman}
%---- Codierung----%
\usepackage[latin1]{inputenc}	% for Unix and Windows
\usepackage[T1]{fontenc}
\usepackage{graphicx}
\usepackage{url}
\usepackage{llncsdoc}
%----- Mathematischer Zeichenvorrat---%
\usepackage{amsmath}
\usepackage{amssymb}
\usepackage{enumerate}
% fuer die aktuelle Zeit
\usepackage{scrtime}
\usepackage{listings}
\usepackage{subfigure}
\usepackage{hyperref}

\usepackage{listings}
\usepackage{tabularx}
\usepackage{color}
\usepackage{colortbl}
\usepackage{graphicx,import}

\setcounter{tocdepth}{3}
\setcounter{secnumdepth}{3}



\begin{document}

\mainmatter
\title{Praktikum Parallele Numerik}
\titlerunning{Arbeit}
\author{Fabian Miltenberger, S�bastien Thill, Thore Mehr}
\authorrunning{Parallele Numerik}
\institute{Betreuer: Markus Hoffmann, Thomas Becker}
\date{23.07.2007}
\maketitle

\begin{abstract}Im Rahmen dieses Praktikums haben wir viel gelernt.
\end{abstract}

\section{Projekt 1}
In diesem Projekt lag der Schwerpunkt auf dem Kennenlernen der Bibliothek \emph{OpenMP} sowie deren Handhabung. Weiter ging es um den \emph{IntelThreadChecker}, ein Programm zum Analysieren von Programmcode auf potentielle Fehler in der Parallelisierung. Zu guter letzt haben wir uns mit der FEM-Methode besch�ftigt, dabei im Speziellen mit dem Gau�-Seidel-Verfahren zum L�sen linearer Gleichungen wie sie bei der Differenzenmethode vorkommen. Zu guter letzt betrachteten wir einige andere Verfahren zum L�sen solcher Probleme und haben das CG-Verfahren implementiert.

\subsection{Aufgabe 1}
\begin{lstlisting}[frame=single, captionpos=b, caption={Beispielhafte Ausgabe des Programms bei Ausf�hrung mit 8 F�den.}, label ={helloworldoutput}, basicstyle=\footnotesize]
Hello World, this is Thread0
Hello World, this is Thread5
Hello World, this is Thread4
Hello World, this is Thread7
Hello World, this is Thread1
Hello World, this is Thread3
Hello World, this is Thread6
Hello World, this is Thread2
\end{lstlisting}
Wie in der Auflistung \ref{helloworldoutput} zu sehen, folgt die Reihenfolge der ausgef�hrten F�den keinem bestimmten Muster. Die Reihenfolgen zwischen verschiedenen Ausf�hrungen sind in der Regel verschieden.


\subsection{Aufgabe 2}



\subsection{Aufgabe 3}
\begin{itemize}
	\item[a)] Im Folgenden einige bekannte Gr��en der Parallelisierung, wie sie in der Vorlesung Rechnerstrukturen gelehrt wurden. \\
	
	Vorab sei $T(n)$ die Ausf�hrungszeit auf n Prozessoren,
	$P(n)$ die Anzahl der auszuf�hrenden Einheitsoperationen und
	$I(n)$ der Parallelindex. \\
	
	Der Speedup/die Beschleunigung $S(n)$:
	\begin{equation}
		S(n) = \frac{T(1)}{T(n)}
	\end{equation}
	
	Die Effizienz $E(n)$:
	\begin{equation}
		E(n) = \frac{S(n)}{n} = \frac{T(1)}{n \cdot T(n)}
	\end{equation}
	
	Der Mehraufwand $R(n)$:
	\begin{equation}
		R(n) = \frac{P(n)}{P(1)}
	\end{equation}
	
	Die Auslastung $U(n)$:
	\begin{equation}
		U(n) = \frac{U(n)}{n} = R(n) \cdot E(n) = \frac{P(n)}{n \cdot T(n)}
	\end{equation}
	
	\item[b)] \begin{description}
		\item[Race Condition] \hfill \\ Wettlaufsituationen
		Dabei h�ngt Ergebnis von konkreter Ausf�hrungsreihenfolge ab (daher Wettlauf)
		Entsteht, wenn verschiedene F�den auf gleiche Variable zugreifen, und mindestens ein Faden deren Wert manipuliert
		Korrektheit der Ergebnisse h�ngt von Ausf�hrungsreihenfolge ab
		\item[Dead lock] \hfill \\ Zyklus im Allokationsgraphen
	\end{description} \hfill

	\item[c)] Die Vor- und Nachteile verschiedener Architekturen bez�glich verschiedener Aspekte:
	\begin{table}
		\begin{center}
			\begin{tabular}{l | p{5cm} p{5cm}}
				& Anwenderfreundlichkeit & Energieeffizienz \\
				\hline
				GPU & Gut & Mittel \\
				CPU & Gut & Gering \\
				FPGA & Gering & Sehr gut \\
				MIC & Gut & Gut
			\end{tabular}
		\end{center}
		\caption{Verschiedene Beschleuniger im Vergleich}
	\end{table}
\end{itemize}

\definecolor{cbblack}{RGB}{220,220,220}
\definecolor{cbwhite}{RGB}{255,255,255}
\subsection{Aufgabe 4}
\begin{itemize}
	\item[a)] Ausf�hrliche Beschreibung der Vorgehensweise... \\
	\item[b)] Geht nicht... \\
	\item[c)] Ausf�hrliche Beschreibung der Vorgehensweise... \\
		Abbildung zu Abh�ngigkeiten
		
		\begin{figure}\centering
			\resizebox{0.5\textwidth}{!}{
			\import{}{gsv_node_dependence.pdf_tex}}
			\caption{Abh�ngigkeiten eines Knoten der Iterierten $i$ zu Nachbarknoten und deren Iterierten.}
		\end{figure}
		
		\begin{figure}
			\begin{center}
				\renewcommand*{\arraystretch}{1.8}
				\begin{tabular}{|c|c|c|c|c|c|c|}
					\hline
					\cellcolor{cbblack}$u^{i+1}_{0,0}$ & \cellcolor{cbwhite}$u^{i+1}_{1,0}$ & \cellcolor{cbblack}$u^{i}_{2,0}$ & \cellcolor{cbwhite}$u^{i}_{3,0}$ & \cellcolor{cbblack}$u^{i-1}_{4,0}$ & \cellcolor{cbwhite}$u^{i-1}_{5,0}$ & \cellcolor{cbblack}$u^{i-2}_{6,0}$ \\
					\hline
					\cellcolor{cbwhite}$u^{i+1}_{0,1}$ & \cellcolor{cbblack}$u^{i}_{1,1}$ & \cellcolor{cbwhite}$u^{i}_{2,1}$ & \cellcolor{cbblack}$u^{i-1}_{3,1}$ & \cellcolor{cbwhite}$u^{i-1}_{4,1}$ & \cellcolor{cbblack}$u^{i-2}_{5,1}$ & \cellcolor{cbwhite}$u^{i-2}_{6,1}$ \\
					\hline
					\cellcolor{cbblack}$u^{i}_{0,2}$ & \cellcolor{cbwhite}$u^{i}_{1,2}$ & \cellcolor{cbblack}$u^{i-1}_{2,2}$ & \cellcolor{cbwhite}$u^{i-1}_{3,2}$ & \cellcolor{cbblack}$u^{i-2}_{4,2}$ & \cellcolor{cbwhite}$u^{i-2}_{5,2}$ & \cellcolor{cbblack}$u^{i-3}_{6,2}$ \\
					\hline
					\cellcolor{cbwhite}$u^{i}_{0,3}$ & \cellcolor{cbblack}$u^{i-1}_{1,3}$ & \cellcolor{cbwhite}$u^{i-1}_{3,3}$ & \cellcolor{cbblack}$u^{i-2}_{3,3}$ & \cellcolor{cbwhite}$u^{i-2}_{5,3}$ & \cellcolor{cbblack}$u^{i-3}_{5,3}$ & \cellcolor{cbwhite}$u^{i-3}_{7,3}$ \\
					\hline
					\cellcolor{cbblack}$u^{i-1}_{0,4}$ & \cellcolor{cbwhite}$u^{i-1}_{1,4}$ & \cellcolor{cbblack}$u^{i-2}_{2,4}$ & \cellcolor{cbwhite}$u^{i-2}_{3,4}$ & \cellcolor{cbblack}$u^{i-3}_{4,4}$ & \cellcolor{cbwhite}$u^{i-3}_{5,4}$ & \cellcolor{cbblack}$u^{i-4}_{6,4}$ \\
					\hline
					\cellcolor{cbwhite}$u^{i-1}_{0,5}$ & \cellcolor{cbblack}$u^{i-2}_{1,5}$ & \cellcolor{cbwhite}$u^{i-2}_{2,5}$ & \cellcolor{cbblack}$u^{i-3}_{3,5}$ & \cellcolor{cbwhite}$u^{i-3}_{4,5}$ & \cellcolor{cbblack}$u^{i-4}_{5,5}$ & \cellcolor{cbwhite}$u^{i-4}_{6,5}$ \\
					\hline
					\cellcolor{cbblack}$u^{i-2}_{0,6}$ & \cellcolor{cbwhite}$u^{i-2}_{1,6}$ & \cellcolor{cbblack}$u^{i-3}_{2,6}$ & \cellcolor{cbwhite}$u^{i-3}_{3,6}$ & \cellcolor{cbblack}$u^{i-4}_{4,6}$ & \cellcolor{cbwhite}$u^{i-4}_{5,6}$ & \cellcolor{cbblack}$u^{i-5}_{6,6}$ \\
					\hline
				\end{tabular}
			\end{center}
			\caption{Verdeutlichung der Vorgehensweise der Parallelisierung. Zuerst werden diejenigen Eintr�ge von $u$ parallel berechnet, die sich in grau Markierten Feldern befinden. Anschlie�end parallel die Eintr�ge in den wei�en Feldern. Es ist zu beachten, dass die Eintr�ge $u_{x,y}$ f�r unterschiedliche Iterierte $i$ berechnet werden.}
		\end{figure}
\end{itemize}

\subsection{Aufgabe 5}
\begin{itemize}
	\item[a)] Die Bedingungen lauten nach den Folien der FEM-Einf�hrung: \\
	$\Omega$ sei ein beschr�nktes Gebiet\\
	$\Gamma$ sei hinreichend glatt\\
	$f:\Omega\rightarrow \mathbb{R}$ gegebene Funktion, wie es hier der Fall ist.\\
	\item[b)] Gesucht ist $f$ mit
	\begin{equation}\label{udef}
		u(x,y)=\sin(2M\pi x)\sin(2N\pi y)
	\end{equation}
	Dies kann durch einfache Anwendung des \emph{Laplace-Operators $\Delta$} berechnet werden:
	\begin{equation}\label{fgleichung}
	\begin{split}
		f(x,y)&=-\Delta u(x,y) \\
		&= -\frac{\partial u}{\partial x^2}-\frac{\partial u}{\partial y^2} \\
		&= -\frac{\partial}{\partial x}(2M\pi\cos(2M\pi x)\sin(2N\pi y)) -\frac{\partial}{\partial y}(2N\pi\sin(2M\pi x)\cos(2N\pi y)) \\
		&= 4M^2\pi^2\sin(2M\pi x)\sin(2N\pi y) + 4N^2\pi^2\sin(2M\pi x)\sin(2N\pi y) \\
		&= (M^2+N^2)4\pi^2\sin(2M\pi x)\sin(2N\pi y)
	\end{split}
	\end{equation}
	\item[c)]Da in \ref{fgleichung} $M,N \in\mathbb{N}$ beliebig, w�hlen wir $M=N=1$ f�r die L�sung dieser Teilaufgabe. Damit ergibt sich
	\begin{equation}
	f(x,y)=8\pi^2\sin(2\pi x)\sin(2\pi y)
	\end{equation}
	Die analytische L�sung $u(x,y)$ ist bereits aus der Aufgabenstellung mit Gleichung \ref{udef} gegeben und wird zur �berpr�fung der Berechnung verwendet.
\end{itemize}

\subsection{Aufgabe 6}
\begin{itemize}
	\item[a)]
	\item[b)]
	\item[c)]
\end{itemize}

\section{Parojekt 2}

% Normaler LNCS Zitierstil
%\bibliographystyle{splncs}
\bibliographystyle{itmalpha}
% TODO: �ndern der folgenden Zeile, damit die .bib-Datei gefunden wird
\bibliography{literatur}

\end{document}

