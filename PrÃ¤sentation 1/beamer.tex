\documentclass[german,notes,18pt]{beamer}
\mode<presentation>
%\usepackage[ngerman, english]{babel}
\usepackage[utf8]{inputenc}
\usepackage[T1]{fontenc}
\usepackage{amsmath,amsfonts,amssymb}
\usepackage{listings}
\usepackage{eurosym}
\usepackage{multirow}
\usepackage{color}
\usepackage{pdfpages}
\usepackage{graphicx}
\usepackage{wasysym}
\graphicspath{{Bilder/}}
%\beamertemplatenavigationsymbolsempty

\usetheme{kit}

\title{Projekt 1}
\subtitle{Arbeiten mit OpenMP und FEM}
\author{Thore Mehr, Fabian Miltenberger, Sébastien Thill}
\date{11.01.2017}
\institute{Lehrstuhl für Rechnerarchitektur und Parallelverarbeitung (ITEC)}

\definecolor{kit}{HTML}{009682}
\definecolor{darkgreen}{RGB}{0, 180, 0}
\definecolor{darkred}{RGB}{180, 0, 0}
\newcommand{\pro}{$\oplus$}
\newcommand{\contra}{$\ominus$}

\definecolor{shinygray}{RGB}{245,245,245}
\lstset{language=Java,
		keywordstyle=\color{blue}\bfseries,
		numberstyle=\color{blue},
		numbers=left,
		tabsize=4,
		xleftmargin=3.5em,
		xrightmargin=2em}

\begin{document}
	\selectlanguage{ngerman}
	
	\frame{\titlepage}
	
	\section{Sektion}
	\subsection{Intro}
	\begin{frame}
		\frametitle{Intro}
		
		\begin{enumerate}
			\item Parallelitätsfehler
			\begin{itemize}
				\item Fehlerursache beschreiben
				\item Testfall Zuverlässigkeit
				\item Zwei Mögliche Lösungen mit Begründung \\
				$\rightarrow$ Begründung und zweite Variante nicht vergessen
			\end{itemize}
		\end{enumerate}
	\end{frame}
	
	\begin{frame}
		\frametitle{Philosophische Überlegungen}
		Es gibt beim parallelen Ablauf Verzögerungen. Manchmal geht es gerade noch gut, manchmal hat man Pech.
		\\
		\vspace{1cm}
		\onslide<2->
		$\hookrightarrow$ Dies war eine Abgabe für Aufgabe 1 b).
	\end{frame}
	
	
	
	\section{Anderes Kapitel}
	\subsection{Unterkapitel}
	\begin{frame}
		\frametitle{Kontrollflussgraphen (KFG)}
		
		Was wollen wir erreichen?
		\begin{itemize}
			\item Programm in Zwischensprache übersetzen
			\item KFG aus Zwischensprache generieren
			\item Anhand KFG Aussagen über Abdeckung treffen
		\end{itemize}
	\end{frame}

\end{document}